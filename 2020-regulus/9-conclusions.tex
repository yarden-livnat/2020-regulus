\section{Conclusions}
\label{sec:conclusions}
The \RT addresses the important, though often neglected, 'why' question by  proposing a new perspective of the topology simplification process. The \RT visualization offers both a concise broad view of the simplification landscape and a guide for an interactive visual exploration of the underlying scalar function. We describe the \RT in the context of Morse-Smale complexes, but the partition's perspective and the tree are equally applicable to Morse complexes. Some of the measures as well as the inverse regression curves are not directly applicable and one will have to use high order regression models.

The \RT has several limitations. First, it does not preserve spatial locality or even adjacency relation between partitions. Mapping spatial locality is a complex issue for multi-dimensional data in general. Adjacency information can be retrieved from the Morse-Smale complexes, though how to depict it is not clear and is especially problematic in a setting with many levels of details. 
Second, Morse-Smale partitions can have complex twisting shapes that are not captured directly in the tree structure and the tree does not address the notion of topological holes. 

We have begun exploring methods for using the inverse regression curves to facilitate adaptive sampling, both for validation purposes and for improving spatial resolutions in areas that are undersampled. We are also looking at using t-SNE and other dimensional reduction and clustering techniques to analyze the linear models and provide additional measures for identifying and highlighting potential unique partitions.

% \section{Acknowledgments}
% This work was funded in part by the U.S. Department of Energy, Nuclear Energy University Program (NEUP) grant DE-NE0008587.
