\abstract{Understanding the response of an output variable to multi-dimensional inputs lies at the heart of many data exploration endeavours. Topology-based methods, in particular Morse theory and persistent homology, provide a useful framework for studying this relationship, as phenomena of interest often appear naturally as fundamental features. The Morse-Smale complex captures a wide range of features by partitioning the domain of a scalar function into piecewise monotonic regions, while persistent homology provides a means to study these features at different scales of simplification. Previous works demonstrated how to compute such a representation and its usefulness to gain insight into multi-dimensional data. However, exploration of the multi-scale nature of the data was limited to selecting a single simplification threshold from a plot of region count. In this paper, we present a novel tree visualization that provides a concise overview of the entire hierarchy of topological features. The structure of the tree provides initial insights in terms of the distribution, size, and stability of all partitions. We use regression analysis to fit linear models in each partition, and develop local and relative measures to further assess uniqueness and the importance of each partition, especially with respect parents/children in the feature hierarchy. The expressiveness of the tree visualization becomes apparent when we encode such measures using colors, and the layout allows an unprecedented level of control over feature selection during exploration. For instance, selecting features from multiple scales of the hierarchy enables a more nuanced exploration. Finally, we demonstrate our approach using examples from several scientific domains.
%
%Morse-Smale theory and persistence homology form a rich and powerful topology-based framework for studying complex high-dimensional scalar functions. A Morse-Smale complex captures a wide range of features by decomposing the scalar function into piecewise monotonic partitions while persistence homology provides a means for simplifying it at different levels of detail. Previous works focused on \textit{how} to create a simplified topological description for a given persistence value but provided only crude measures to help a user select an appropriate value from  ordered list of persistence values. 
%In this paper, we present a novel tree visualization that provides a concise hierarchical overview of all simplification steps with respect to their persistence levels. The overall structure of the tree provides initial insights in terms of the distribution, size and stability of all potential space partitioning. Rather than relying only on topological attributes, we use regression analysis to fit linear models in all potential partitions and develop local and relative measures to further assess uniqueness and the importance of each partition. The expressiveness of the tree visualization becomes apparent when we encode measures using colors and interact with one or more trees to guide the exploration of simplification space and the underlying function, and identify partitions of unique behavior. In addition, we discuss the notions of local and non-consistent simplifications using partitions from different persistence levels. Finally, we demonstrate our approach using examples from several scientific domains.
} 
