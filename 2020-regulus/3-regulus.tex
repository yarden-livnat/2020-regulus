\section{Regulus}
\label{sec:motivation}

While topological structures such as contour trees, merge trees and Morse-Smale complexes 
%are based on the function value at the critical points. It is important to note that while they do 
can capture features at multiple scales,
%and are often used to facilitate progressive simplifications, 
they nevertheless do not describe the simplification process, nor do they provide an overview of all the simplified topologies; rather, each instance describes, and is used to explore, only one simplified topology. Conceptually, simplification consist of creating a series of progressively coarse variation of one of these topological structures. In practice, only one model is created and then transformed to describe the required simplification level.
Visual exploration methods~\cite{Gerber10, maljovec16} are also designed to visualize one simplified topology at a time. Often, phenomena of interest appear at different scales in the data, and a single simplification threshold is insufficient for analysis.
The question of why should the user select a \textit{particular} simplification level was mostly left to the user's best estimation. 

In this work, we focus on the 'why' question in the context of using multi-dimensional Morse-Smale complexes  to study the relationships between input parameters and the output function. Rather than develop a method to find an optimal simplification threshold, our approach is to develop a visual representation of the whole persistence space that can help guide the user exploration. The new visualization, called Regulus Tree, is based on an interpretation of the simplification process in terms of nested partitions rather than cancellation on critical points. The expressiveness of the \RT comes to light when various attributes and measures are encoded on top of it. Another consideration of our design is to empower users to define their own attributes and measures and enable on the fly modification. The \RT enables,
% \begin{itemize}[itemsep=0pt]

    \noindent \textbf{Noise:} identify regions where noise is prominent
    
    \noindent  \textbf{Persistence level:} gain better understanding of the plateaus in terms of the size and stability of the partitions involves. Compare the statistical characteristics on the set of partitions for different persistence levels
    
    \noindent  \textbf{Adaptive simplification}: Select multiple persistence levels for individual features to adapt the simplification based on amount of relative rather than absolute noise, adapt to the local scale of features, and other measures of interest
    
    \noindent  \textbf{Local properties:} Compute and display local attributes of the function in different regions
    
    \noindent  \textbf{Relative measures:} Compare and contrast partitions from different locations in the function space as well as from different levels of details (persistence levels)
    
    \noindent  \textbf{Uniqueness:} Identify and study partitions that exhibit unique characteristics
    
    \noindent  \textbf{Clarity:} The \RT provide a hierarchical view of the persistence space in terms of nested partitions, which our collaborator scientists found much easier to grasp and comprehend as opposed to the technical description in terms of critical points cancellations. 
% \end{itemize}

In the following, we present the conceptual design, structure, and layout of the {\RT} as well as ways to simplify the tree itself. We describe several ways the \RT can be used for various tasks along with additional supporting views. We then introduce the notion of dynamic attributes and measures and show how they can provide unique insights and help guide the user exploration.
